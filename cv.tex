%%%%%%%%%%%%%%%%%%%%%%%%%%%%%%%%%%%%%%%%%
% "ModernCV" CV and Cover Letter
% LaTeX Template
% Version 1.11 (19/6/14)
%
% This template has been downloaded from:
% http://www.LaTeXTemplates.com
%
% Original author:
% Xavier Danaux (xdanaux@gmail.com)
%
% License:
% CC BY-NC-SA 3.0 (http://creativecommons.org/licenses/by-nc-sa/3.0/)
%
% Important note:
% This template requires the moderncv.cls and .sty files to be in the same 
% directory as this .tex file. These files provide the resume style and themes 
% used for structuring the document.
%
%%%%%%%%%%%%%%%%%%%%%%%%%%%%%%%%%%%%%%%%%

%----------------------------------------------------------------------------------------
%	PACKAGES AND OTHER DOCUMENT CONFIGURATIONS
%----------------------------------------------------------------------------------------



\documentclass[11pt,a4paper,sans]{moderncv} % Font sizes: 10, 11, or 12; paper sizes: a4paper, letterpaper, a5paper, legalpaper, executivepaper or landscape; font families: sans or roman

\moderncvstyle{casual	} % CV theme - options include: 'casual' (default), 'classic', 'oldstyle' and 'banking'
\moderncvcolor{blue} % CV color - options include: 'blue' (default), 'orange', 'green', 'red', 'purple', 'grey' and 'black'

\usepackage[scale=0.75]{geometry} % Reduce document margins
%\setlength{\hintscolumnwidth}{3cm} % Uncomment to change the width of the dates column
%\setlength{\makecvtitlenamewidth}{10cm} % For the 'classic' style, uncomment to adjust the width of the space allocated to your name

\usepackage{comment}

%----------------------------------------------------------------------------------------
%	NAME AND CONTACT INFORMATION SECTION
%----------------------------------------------------------------------------------------

\firstname{Emmanuel} % Your first name
\familyname{Senft} % Your last name

% All information in this block is optional, comment out any lines you don't need
\title{Curriculum Vitae}

\begin{document}

\makecvtitle % Print the CV title

%----------------------------------------------------------------------------------------
%	Contact SECTION
%----------------------------------------------------------------------------------------

\section{CONTACT}

\cvitem{Address}{Plymouth University, A216 Portland Square}
\cvitem{}{Drake Circus, Plymouth, PL4 8AA}
\cvitem{}{United Kingdom}
\cvitem{Email}{emmanuel.senft@plymouth.ac.uk}
\cvitem{Phone}{+44 (0)7751 689105}
\cvitem{Website}{\url{https://emmanuel-senft.github.io/index.html}}
\cvitem{Scholar}{\url{https://scholar.google.co.uk/citations?user=Yw1tGf4AAAAJ&hl=en}}

%----------------------------------------------------------------------------------------
%	Statement of objectives SECTION
%----------------------------------------------------------------------------------------

\section{RESEARCH STATEMENT}

\cvitem{Robots Learning to Interact}{My research is focused on exploring ways to allow robots to learn complex policies in the real world, such as behaving socially in a Human-Robot Interaction. 
Learning by interacting in the real world is a challenging topic, because classical approaches such as Reinforcement Learning or learning from big data can not be used because of their dependence on random exploration (leading to negative impact on the world) or the lack of large dataset for real world interaction. Part of the DREAM project, I am using Robot-Assisted Therapy and robots in education as scenarios to provide realistic situations, constraints and problems to guide the design and the evaluation SPARC, the interaction framework used to combine human control and robot learning.
}

%----------------------------------------------------------------------------------------
%	EDUCATION SECTION
%----------------------------------------------------------------------------------------

\section{EDUCATION}
\cventry{2014-present}{PhD}{Plymouth University}{Plymouth, UK}{Title: Teaching Robots Social Autonomy From In Situ Human Supervision}{Supervisory team: Professor Tony Belpaeme, Dr. Paul Baxter and Dr S\'{e}verin Lemaignan}
\cventry{2011--2013}{Master of Science in Robotic and Autonomous Systems}{Swiss Federal Institute of Technology Lausanne (EPFL)}{Lausanne, Switzerland}{\textit{Grade 5.5/6}}{With Minor in Area and Cultural Studies.}  % Arguments not required can be left empty
\cventry{2008--2011}{Bachelor of Science in Microengineering}{Swiss Federal Institute of Technology Lausanne (EPFL)}{Lausanne, Switzerland}{\textit{Grade 5.29/6}}{}

\section{MASTER'S THESIS}

\cvitem{Title}{\emph{Hybrid Optimisation with Roombots} (Maximum possible grade obtained: 6/6)}
\cvitem{Professor}{Auke J. Ijspeert}

%----------------------------------------------------------------------------------------
%	WORK EXPERIENCE SECTION
%----------------------------------------------------------------------------------------
\section{SELECTED PUBLICATIONS}

\cvitem{2017}{\textbf{Supervised Autonomy for Online Learning in Human-Robot Interaction }
 - \textbf{E. Senft}, P. Baxter, J. Kennedy, S. Lemaignan and T. Belpaeme
 - Pattern Recognition Letters, 2017.}

\cvitem{}{\textbf{How to build a supervised autonomous system for
robot-enhanced therapy for children with autism spectrum disorder}
 - P.G. Esteban, ... , \textbf{E. Senft}, ... , T.Ziemke
 - Paladyn, Journal of Behavioral Robotics. De Gruyter Open, 2017.}

\cvitem{2016}{\textbf{From Characterising Three Years of HRI to Methodology and Reporting Recommendations}
 - P. Baxter, J. Kennedy, \textbf{E. Senft}, S. Lemaignan and T. Belpaeme 
 - Proc. 11th ACM/IEEE International Conference on Human-Robot Interaction (alt.HRI) (HRI), New Zealand, Christchurch, March 2016.}

\cvitem{}{\textbf{Social Robot Tutoring for Child Second Language Learning }
 - J. Kennedy, P. Baxter, \textbf{E. Senft} and T. Belpaeme 
 - Proc. 11th ACM/IEEE International Conference on Human-Robot Interaction (HRI), New Zealand, Christchurch, March 2016.}

\cvitem{2015}{\textbf{SPARC: Supervised Progressively Autonomous Robot Competencies}
 - \textbf{E. Senft}, P. Baxter, J. Kennedy and T. Belpaeme.
 - Proc. 7th International Conference of Social Robotics (ICSR), Paris, France, October 2015.}

\cvitem{}{\textbf{Human-Guided Learning of Social Action Selection for Robot-Assisted Therapy}
 - \textbf{E. Senft}, P. Baxter and T. Belpaeme. 
 - Proc. 4th Workshop on Machine Learning for Interactive Systems (MLIS), Lille, France, July 2015.}

\cvitem{2013}{\textbf{An experimental study on the role of compliant elements on the locomotion of the self-reconfigurable modular robots Roombots.}
 - Vespignani, \textbf{E. Senft}, S. Bonardi, R. Möckel and A. Ijspeert. 
 - Proc. IEEE/RSJ International Conference on Intelligent Robots and Systems (IROS), Tokyo, Japan, November 2013.}

\cvitem{Complete list with pdf}{\url{https://emmanuel-senft.github.io/publication.html}}

\section{PROFESSIONAL EXPERIENCE}
\cventry{since 2015}{Reviewer}{Scientific Reports, JIST, ICRA, SBLI, MECH, IBERDISCAP, HAI, JHRI, IS, HRI, ICSR, HUMANOID, RO-MAN, TAROS.}{}{}{}

\cventry{2015-2017}{Demonstrator}{ROCO 318 - Mobile and Humanoid Robots}{Plymouth University, UK}{}{Leading practical work on bipedal gait design and control for 3rd year students.}

\cventry{2014}{Construction Assistant}{Biorobotic Lab, EPFL}{Lausanne, Switzerland}{Two months}{Development of a controller using a camera and an IMU as inputs to allow autonomous docking on a passive grid using Roombots.}

\cventry{2012--2013}{Internship}{EPFL+ECAL Lab}{Lausanne, Switzerland}{Four months}{Development of a driver for a multitouch tactile wall for a permanent interactive installation at the Red Cross Museum, Geneva, Switzerland.}

%\cventry{2012}{Minor Thesis}{Humanity College, EPFL}{Lausanne, Switzerland}{}{Robotic Development in Japan, Impact and Role in the Society.}

\pagebreak
%----------------------------------------------------------------------------------------
%	COMPUTER SKILLS SECTION
%----------------------------------------------------------------------------------------

\section{TECHNICAL SKILLS}

\cvitem{Advanced}{Python, ROS, QML, C++, Machine Learning (Neural Networks and
Reinforcement Learning), Robotics, Nao, Human-Robot Interaction, Study Design, Academic Writing}
\cvitem{Intermediate}{Solidworks, ProEngineer, MatLab, C, JAVA, Pepper and YARP}
\cvitem{Basic}{Robot Simulator (Webots) and Assembly}

%----------------------------------------------------------------------------------------
%	COMMUNICATION SKILLS SECTION
%----------------------------------------------------------------------------------------
\begin{comment}
\section{Communication Skills}

\cvitem{2010}{Oral Presentation at the California Business Conference}
\cvitem{2009}{Poster at the Annual Business Conference in Oregon}
\end{comment}
%----------------------------------------------------------------------------------------
%	LANGUAGES SECTION
%----------------------------------------------------------------------------------------

\section{Languages}

\cvitemwithcomment{French}{Mother tongue}{}
\cvitemwithcomment{English}{Advanced}{Conversationally fluent and technical}
\cvitemwithcomment{German}{Basic}{Level A2}
\cvitemwithcomment{Italian}{Elementary}{Level A1}
\cvitemwithcomment{Mandarin}{Elementary}{Level A1}

%----------------------------------------------------------------------------------------
%	INTERESTS SECTION
%----------------------------------------------------------------------------------------

\section{OTHER EXPERIENCE}
\cventry{2016-2017}{Comittee member}{Karate-UPSU}{Plymouth University, UK}{}{Treasurer for the university karate club.}
\cventry{2014-2018}{Member}{A+E}{Plymouth University, UK}{}{Mountaineering, climbing and caving student club.}
\cventry{2014-2018}{Member}{Karate-UPSU}{Plymouth University, UK}{}{Karate student club.}
\cventry{2010-2013}{Comittee member}{Jdr-Poly}{EPFL}{}{Association for the promotion of Role Playing Games and Board Games on the campus.}
\cventry{2007-2013}{Scout Team Leader}{Eclaireuses et Eclaireurs Unionistes de France}{Douvaine, France}{}{Team management and leadership, organisation of events with 8-12 or 12-16 years old children with legal responsibility, member since 1998.}
\cventry{2008-2012}{Member}{Robopoly}{EPFL}{}{Robotic Association.}

%\section{PERSONAL INFORMATION}
%\cvitem{Date of Birth}{18/03/1991}
%\cvitem{Nationalily}{French}
%\cvitem{Passport number}{14AD34461}

\begin{comment}
\section{Interests}

\renewcommand{\listitemsymbol}{-~} % Changes the symbol used for lists

\cvlistdoubleitem{Piano}{Chess}
\cvlistdoubleitem{Cooking}{Dancing}
\cvlistitem{Running}
\end{comment}
\end{document}
